\documentclass{article}
\usepackage{blindtext}
\usepackage[a4paper, total={18cm, 25cm}]{geometry}
\usepackage{amsmath}
\usepackage{amsthm}
\usepackage{amssymb}
\usepackage{tabularx}
\usepackage{mathtools}
\usepackage{booktabs}
\usepackage[czech]{babel}
\usepackage[useregional=numeric]{datetime2}
\usepackage{graphicx}
\usepackage{paralist}
\usepackage{pict2e}
\usepackage[dvipsnames]{xcolor}
\usepackage{setspace}

% Theorems settings
\newcommand{\limit}[1]{\lim_{n\rightarrow\infty}#1}
\newtheorem{example}{Cv}[section]
\theoremstyle{definition}
% Diameter symbol settings
\DeclareRobustCommand{\slashcirc}{{\mathpalette\doslashcirc\relax}}
\makeatletter
\newcommand{\doslashcirc}[2]{%
\sbox\z@{$#1\m@th\circ$}%
\setlength\unitlength{\wd\z@}
\begin{picture}(1,1)
	\roundcap
	\put(0,0){\box\z@}
	\put(0,0){\line(1,1){1}}
  \end{picture}%
}
\makeatother
\newcommand{\diameter}[2]{$\slashcirc #1$#2}
%Bold text tmp solution
\newcommand{\voc}[1]{\textbf{#1}}
% TODO command for memory
\newcommand{\TODO}[1]{
	\begin{large}
		\berry{
			(\textbf{TODO:}#1)
		}
	\end{large}
}
% Math symbols
\newcommand{\R}{\mathbb{R}}
\newcommand{\C}{\mathbb{C}}
\newcommand{\N}{\mathbb{N}}
\newcommand{\Q}{\mathbb{Q}}
\newcommand{\Z}{\mathbb{Z}}
\def\doubleunderline#1{\underline{\underline{#1}}}
%Text colors setting
\newcommand{\red}[1]{\textcolor{Red}{#1}}
\newcommand{\blue}[1]{\textcolor{Blue}{#1}}
\newcommand{\green}[1]{\textcolor{Green}{#1}}
\newcommand{\brown}[1]{\textcolor{Brown}{#1}}
\newcommand{\yellow}[1]{\textcolor{Yellow}{#1}}
\newcommand{\magenta}[1]{\textcolor{Magenta}{#1}}
\newcommand{\lime}[1]{\textcolor{LimeGreen}{#1}}
\newcommand{\violet}[1]{\textcolor{Violet}{#1}}
\newcommand{\orange}[1]{\textcolor{Orange}{#1}}
\newcommand{\purple}[1]{\textcolor{Purple}{#1}}
\newcommand{\brick}[1]{\textcolor{BrickRed}{#1}}
\newcommand{\berry}[1]{\textcolor{WildStrawberry}{#1}}
% page command for new page easy use
\newcommand{\page}{\pagebreak}
% List items shortcuts
\newcommand{\bitem}[1]{\item \textbf{#1}} %bold item command
\newcommand{\mitem}[1]{\item \textbf{\large{#1}}} %main item command (large bold item)
\newcommand{\dtitem}[1]{\item[•] #1} %dot item
\newcommand{\dsitem}[1]{\item[-] #1} %dash item
\newcommand{\litem}[1]{\item \begin{large}#1\end{large}} %large item
% command for table of content but with new page command and no page numbering
\newcommand{\content}{\tableofcontents
    \thispagestyle{empty}
    \page}
% Section numbering to arabic
\renewcommand*\thesection{\arabic{section}}
\renewenvironment{itemize}[1]{\begin{compactitem}#1}{\end{compactitem}}
\renewenvironment{enumerate}[1]{\begin{compactenum}#1}{\end{compactenum}}
\renewenvironment{description}[0]{\begin{compactdesc}}{\end{compactdesc}}
\DeclarePairedDelimiter\abs{\lvert}{\rvert}
% Image store location
\graphicspath{{../images/}}

\begin{document}
  \section*{Teoretické základy informatiky 1}

  \subsection*{Predikátový jazyk (P)}
  \begin{enumerate}
    \item  Symboly pro konstanty
    \item Symboly relační - predikátové $=$
    \item Funkční symboly $f$
    \item symboly pro proměnné
    \item symboly pro logické spojky $\neg, \lor, \land, \implies, \impliedby, \iff$
    \item symboly pro kvantifikátory $\forall, \exists$
    \item symboly pomocné $\left(\right), [] , \{\},;$
  \end{enumerate}
 

  \subsection*{Slovo}
  \begin{enumerate}
    \item Každá konstanta
    \item Každá proměnná je term
    \item Jsou-li $t_1, \dots , t_n$ termy P-jazyka, a je-li $f$ funkční symbol P-jazyka $f(t_1, \dots, t_n)$ je term
  \end{enumerate}

  \subsection*{Formule P-jazyka budeme nazývat}
  \begin{enumerate}
    \item Každou atomickou formuly
    \item Jsou-li $\psi, \varphi$ formule, pak také $(\psi \land \varphi), (\psi \lor \varphi), (\psi \implies \varphi), (\psi \iff \varphi), \neg \varphi$ jsou formile
    \item Jetliže $x$ je proměnná a $\varphi$ je formule, potom také $(\forall x)\varphi$ a $(\exists x)\varphi$ jsou formule
    \item jiné formule P-jazyk nemá
  \end{enumerate}

  \subsection*{Syntaktický strom}
  \begin{align*}
    (\forall x)(\forall y)(\forall z)(x<z  &\implies x+z=y+z)
    \\(\forall y)(\forall z)(x<z  &\implies x+z=y+z)
    \\(\forall z)(x<z  &\implies x+z=y+z)
    \\x<z  &\implies x+z=y+z
\end{align*}

\begin{description}
    \item Výskyt proměnné $x$ ve formuli $\varphi$ nazýváme vázaným právě tehdy, když na cestě od libovolného listu tohoto syntetického stromu k základní formuli se objevý $(\forall x)$ nebo $(\exists x)$ V opačném případě nazýváme výskyt $t$ podstatné volný
\end{description}

\subsection*{Predikátový počet Te Mno}
\begin{enumerate}
    \item Konstanty - $0,1, \O, U$
    \item Predikáty - $= (\equiv), \in, \subset$
\end{enumerate}
\begin{align*}
    x \subseteq y &=_{df} (\forall u ) (u \in x \implies u \in y)
    \\x \subset y &=_{df} x \subseteq y \land \neg(x=y)
\end{align*}
\begin{description}
    \item Možina je soubor určitých a rozlišitelných objektů. Tyto objekty nezýváme prvky dané množiny. $J$
\end{description}
\begin{align*}
    J &\dots p \dots p \in J \\
     &\land \\
     p \in J &\dots p \notin J
\end{align*}
\subsection*{Vennovy diagramy}
\end{document}
