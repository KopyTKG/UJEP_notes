\documentclass{article}
\usepackage{blindtext}
\usepackage[a4paper, total={18cm, 25cm}]{geometry}

% METH !!!!
\usepackage{amsmath}
\usepackage{amsthm}
\usepackage{amssymb}

\theoremstyle{definition}
\newtheorem{example}{Cv}[section]

% ----
\usepackage{tabularx}
\usepackage{mathtools}
\usepackage{booktabs}
\usepackage[czech]{babel}
\usepackage[useregional=numeric]{datetime2}
\usepackage{graphicx}
\usepackage{paralist}
\usepackage{pict2e}
\usepackage[dvipsnames]{xcolor}
\usepackage{setspace}

\DeclareRobustCommand{\slashcirc}{{\mathpalette\doslashcirc\relax}}

\makeatletter
\newcommand\doslashcirc[2]{%
  \sbox\z@{$#1\m@th\circ$}%
  \setlength\unitlength{\wd\z@}
  \begin{picture}(1,1)
  \roundcap
  \put(0,0){\box\z@}
  \put(0,0){\line(1,1){1}}
  \end{picture}%
}
\makeatother

\newcommand{\diameter}{$\slashcirc$ }

\newcommand{\TODO}[1]{
	\begin{large}
		\berry{
			(\textbf{TODO:}#1)
		}
	\end{large}
}

\def\doubleunderline#1{\underline{\underline{#1}}}

\newcommand{\red}[1]{\textcolor{Red}{#1}}
\newcommand{\blue}[1]{\textcolor{Blue}{#1}}
\newcommand{\green}[1]{\textcolor{Green}{#1}}
\newcommand{\brown}[1]{\textcolor{Brown}{#1}}
\newcommand{\yellow}[1]{\textcolor{Yellow}{#1}}
\newcommand{\magenta}[1]{\textcolor{Magenta}{#1}}
\newcommand{\lime}[1]{\textcolor{LimeGreen}{#1}}
\newcommand{\violet}[1]{\textcolor{Violet}{#1}}
\newcommand{\orange}[1]{\textcolor{Orange}{#1}}
\newcommand{\purple}[1]{\textcolor{Purple}{#1}}
\newcommand{\brick}[1]{\textcolor{BrickRed}{#1}}
\newcommand{\berry}[1]{\textcolor{WildStrawberry}{#1}}

% page command for new page easy use
\newcommand\page{\newpage}
% new paragraph without bold text
\newcommand{\para}[1]{
	\paragraph{\normalfont{#1}}
}

\newcommand{\bitem}[1]{\item \textbf{#1}} %bold item command
\newcommand{\mitem}[1]{\item \textbf{\large{#1}}} %main item command (large bold item)
\newcommand{\dtitem}[1]{\item[•] #1} %dot item
\newcommand{\dsitem}[1]{\item[-] #1} %dash item
\newcommand{\litem}[1]{\item \begin{large}
#1
\end{large}} %large item

% command for table of content but with new page command and no page numbering
\newcommand{\content}{\tableofcontents
    \thispagestyle{empty}
    \page}

% Section numbering to arabic
\renewcommand*\thesection{\arabic{section}}
\renewenvironment{itemize}[1]{\begin{compactitem}#1}{\end{compactitem}}
\renewenvironment{enumerate}[1]{\begin{compactenum}#1}{\end{compactenum}}
\renewenvironment{description}[0]{\begin{compactdesc}}{\end{compactdesc}}
\DeclarePairedDelimiter\abs{\lvert}{\rvert}
\newcommand{\frontpage}[2]{
\begin{center}
       \vspace{1cm}
       \begin{center}
       		\begin{small}
               Vyšší odborná škola ekonomická, sociální a zdravotnická, \break
		Obchodní akademie, Střední pedagogická škola a Střední zdravotnická škola, 
						Most, příspěvková organizace 
       		\end{small}
       \end{center}
		\vspace{\fill}
        \begin{center}
        		\begin{Huge}
        			\textbf{Seminární práce}
        		\end{Huge}
        	\end{center}
        	\begin{center}
        		\begin{Large}
        			\textbf{Sociální podnikání}
        		\end{Large}
        \end{center}
        %\vspace{16cm}
        \vspace{\fill}
        \begin{normalsize}
        		\begin{tabular}{p{0cm} p{1.5cm} p{10cm}}
        		& \textbf{Jméno:} & Martin Kopecký   \\
        		& \textbf{Třída:} & 1.R    \\
       	 	& \textbf{Obor:}  & Firemní ekonomika \\
        		& \textbf{Datum:} & \Today
        		\end{tabular}
        \end{normalsize}
    \end{center}
    \thispagestyle{empty}
    \page
}



\graphicspath{{../images/}}
\begin{document}
 \section*{Repetitorium matematiky}
\(
(a+b)^2 = a^2 +2ab + b^2 \\
(a-b)^2 = a^2 -2ab + b^2 \\
(a+b)(a-b) = a^2 - b^2 \\\\
D = b^2 - 4ac \\\\
x_{1,2} = \frac{-b \pm \sqrt{D}}{2a} \\ \\
a = \log_b{x} \Rrightarrow b^a = x\\
\log{a} + \log{b} = \log{a\cdot b} \\
\log{a} - \log{b} = \log{\frac{a}{b}} \\
\log_a{b^n} = n\cdot\log_a{b} \\
a^{\log_a{b}} = b \\
\frac{\log{a}}{\log{b}} = \log_b{a} \\
\log{(3+x)} = 0 \rightarrow 3+x = 1
\)
\page
\subsection*{Test repetitorium} 
\begin{example}
    Upravte do základního tvaru: $\left(
        \frac{-16}{3}
    \right) \cdot 
    \sqrt{
        \frac{2}{3}} + 
        \frac{\frac{64}{9}}
        {2\cdot\sqrt{\frac{2}{3}}}$
\end{example}
\begin{gather*}
    \left(
        \frac{-16}{3}
    \right) \cdot 
    \sqrt{
        \frac{2}{3}} + 
        \frac{\frac{64}{9}}
        {2\cdot\sqrt{\frac{2}{3}}} = 
         - \frac{2^4}{3} \cdot \frac{2^\frac{1}{2}}{3^\frac{1}{2}} + \frac{\frac{2^6}{3^2}}{2 \cdot \frac{2^\frac{1}{2}}{3^\frac{1}{2}}} 
        = - \frac{2^\frac{9}{2}}{3^\frac{3}{2}} + \left( \frac{2^6}{3^2} \cdot \frac{3^\frac{1}{2}}{2^\frac{3}{2}} \right)
        = - \frac{2^\frac{9}{2}}{3^\frac{3}{2}} + \frac{2^\frac{9}{2}}{3^\frac{3}{2}} = \doubleunderline{0}
\end{gather*}
\begin{example}
    Zjednodušte $\biggl( x-\frac{3x}{x+1} \biggr)\cdot 
    \biggl( \frac{x-1}{x-2} - \frac{x}{x-1} \biggr) $
\end{example}
\begin{gather*}
    \biggl( x-\frac{3x}{x+1} \biggr)\cdot 
    \biggl( \frac{x-1}{x-2} - \frac{x}{x-1} \biggr) = 
    \biggl( x-\frac{3x}{x+1} \biggr)\cdot 
    \biggl(\frac{x-1}{x-1} \cdot \frac{x-1}{x-2} - \frac{x}{x-1} \biggr) = \\
     = \biggl( x-\frac{3x}{x+1} \biggr)\cdot 
    \biggl( \frac{(x-1)^2}{(x-2)(x-1)} - \frac{x}{x-1} \biggr) = 
    \biggl( x-\frac{3x}{x+1} \biggr)\cdot 
    \biggl( \frac{(x-1)^2 - \bigl(x \cdot (x-2)\bigr)}{(x-2)(x-1)}\biggr) =
    \\  
    = \biggl(\frac{(x+1)x}{x+1} -\frac{3x}{x+1} \biggr)\cdot 
    \biggl( \frac{(x-1)^2 - \bigl(x^2-2x\bigr)}{(x-2)(x-1)}\biggr) =
    \\
    = \biggl(
        \frac{x^2-2x}{x+1} 
    \biggr) \cdot 
    \biggl(
        \frac{x^2-2x+1-x^2+2x}{(x-2)(x-1)}
    \biggr) = \biggl(
        \frac{x^2-2x}{x+1}
    \biggr) \cdot \biggl(
        \frac{1}{(x-2)(x-1)}
    \biggr) = \\=   \frac{x(x-2)}{x+1}  \cdot \frac{1}{(x-2)(x-1)} = \frac{x}{x+1} \cdot \frac{1}{x-1} = \doubleunderline{\frac{x}{x^2-1}}
\end{gather*}
\begin{example}
    Zjednodušte $\frac{3y+2}{y^2-2y+1} - \frac{6}{y^2-1} - \frac{3y-2}{y^2+2y+1}$
\end{example} 
\begin{align*}
    \frac{3y+2}{y^2-2y+1} - \frac{6}{y^2-1} - \frac{3y-2}{y^2+2y+1} 
     &= \frac{3y+2}{(y-1)(y-1)} - \frac{6}{(y+1)(y-1)} - \frac{3y+2}{(y+1)(y+1)} 
    \\ &= \frac{((3y+2)\cdot(y+1))-((6)\cdot(y-1))}{(y-1)^2(y+1)} - \frac{3y+2}{(y+1)^2}
    \\ &= \frac{3y^2+3y+2y+2-6y+6}{(y-1)^2(y+1)} - \frac{3y+2}{(y+1)^2}
    \\ &= \frac{3y^2-y+8}{(y-1)^2(y+1)} - \frac{3y+2}{(y+1)^2}
    \\ &= \frac{((3y^2-y+8)\cdot(y+1))-((3y+2)(y^2-2y+1)}{(y^2-1)^2}
    \\ &= \frac{(3y^3+3y^2-y^2-y+8y+8)-(3y^3-6y^2+3y+2y^2-4y+2)}{(y^2-1)^2}
    \\ &= \frac{3y^3+3y^2-y^2-y+8y+8-3y^3+6y^2-3y-2y^2+4y-2}{(y^2-1)^2}
    \\ &= \frac{6y^2+8y+6}{(y^2-1)^2}
    = \doubleunderline{\frac{2(3y^2+4y+3)}{(y^2-1)^2}}
\end{align*}
\begin{example}
    Řešte rovnici s neznámou $x \in \mathbb{R} $ 
\end{example}
\begin{align*}
    \frac{x+5}{10} - \frac{x-4}{8} &= 1
    \\ \frac{((x+5)\cdot(8)) - ((x-4)\cdot(10))}{80} &= 1
    \\ \frac{8x+40-10x+40}{80} &= 1
    \\ -2x + 80 &= 80
    \\ -2x &= 0
    \\ x &= \doubleunderline{0}
\end{align*}
\begin{example}
    Řešte soustavu dvou rovnic o dvou neznámých $x$,$y \in \mathbb{R}$
\end{example}
\begin{align*}
    3x &= -4y+1
    \\ 3y &= 4x
    \\ y &= \frac{4x}{3}
    \\ 3x &= -\frac{16x}{3} + 3
    \\ 9x &= -16x +3
    \\ 25x &= 3
    \\  x &= \doubleunderline{\frac{3}{25}}
    \\ y &= \frac{\frac{4}{1}\cdot \frac{3}{25}}{\frac{3}{1}} = \frac{12}{25} \cdot \frac{1}{3} = \frac{12}{75} = \doubleunderline{\frac{4}{25}}
\end{align*}
\begin{example}
    Řešte nerovnici s neznámou $x \in \R$
\end{example}
\begin{align*}
    x^2+10 &> 7x
    \\
    x^2 -7x + 10 &> 0 \\
    D = (-7)^2 - 4\cdot1\cdot10 &= 49-40 = 9 \\
    x_{1,2} = \frac{7 \pm 3}{2} &\Rrightarrow x_1 = 5, x_2 = 2 \\\\
    x \in (-\infty, 2) &\cup (5, \infty) 
\end{align*}
\begin{example}
    Řešte rovnici s neznámou $x \in \R$
\end{example}
\begin{align*}
    2^x - \frac{1}{2} &= \frac{1}{2} - 2^x
    \\ 2 \cdot 2^x &= 1
    \\ 2^{x+1} &= 2^0
    \\ x+1 &= 0 
    \\ x &= \doubleunderline{-1} 
\end{align*}
\begin{example}
    Určete definiční obor funkce $f(x)$
\end{example}
\begin{align*}
    f(x) &= \frac{\log{(2-x)}}{\log{(x+3)}}
    \\ \\ \log{(x+3)} &= 0  
    \\ \\ x &\neq \{-3,2\}
\end{align*}
\page
\subsection*{Hodina 07.10.2022}
 \begin{gather*}
    \biggl( \frac{1}{x+1} - \frac{2x}{x^2-1} \biggr) \biggl(\frac{1}{x} - 1\biggr)
    =  \biggl( \frac{1}{ x+1} - \frac{2x}{(x+1)(x-1)}\biggr)\biggl(\frac{1}{x} -1 \biggr)
    = \\
    = \biggl( \frac{1}{x+1} \frac{x-1}{x-1} - \frac{2x}{(x+1)(x-1)} \biggr)  \biggl(\frac{1}{x}-1\biggr)
    = \biggl( \frac{x-1}{(x+1)(x-1)} - \frac{2x}{(x+1)(x-1)}\biggr) \biggl(\frac{1}{x} - 1\biggr) = 
    \\
    = \biggl(\frac{-x-1}{(x+1)(x-1)}\biggr)\biggl(\frac{1}{x} - \frac{x}{x}\biggr) 
    = \frac{-x-1}{(x+1)(x-1)} \cdot \frac{1 - x}{x}
    = \frac{-(x+1)}{(x+1)(x-1)} \cdot \frac{1-x}{x} =
    \\ = \frac{-1}{x-1} * \frac{1-x}{x} = \frac{-(1-x)}{(x-1)(x)} = \frac{-(-1)(x-1)}{(x-1)x} = \doubleunderline{\frac{1}{x}}
\end{gather*}
\begin{gather*}
    \biggl( \frac{3}{(x-3)^2} + \frac{1}{x+3}- \frac{6}{x^2-9}\biggr) \cdot \frac{x^2-6x+9}{2}
    = \biggl( \frac{3}{(x-3)(x-3)} + \frac{1}{x+3} - \frac{6}{(x+3)(x-3)} \biggr) \cdot \frac{(x-3)(x-3)}{2} =
    \\ = \biggl(
    \frac{3}{(x-3)(x-3)} + \frac{x-3}{(x+3)(x-3)} - \frac{6}{(x+3)(x-3)}
    \biggr) \cdot \frac{(x-3)(x-3)}{2} = \\ = 
    \biggl(
        \frac{3}{(x-3)(x-3)} + \frac{x-9}{(x+3)(x-3)}
    \biggr) \cdot \frac{(x-3)(x-3)}{2} =
    \\ = \biggl(
        \frac{3}{(x-3)(x-3)} + \frac{x-9}{(x+3)(x-3)} \cdot \frac{(x-3)}{(x-3)}
    \biggr) \cdot \frac{(x-3)(x-3)}{2} = \\ =
     \biggl(
        \frac{3}{(x-3)^2} + \frac{(x-9)(x-3)}{(x+3)(x-3)^2} 
    \biggr) \cdot \frac{(x-3)^2}{2} = \frac{3x+9+\bigl((x-9)(x-3)\bigr)}{(x+3)(x-3)^2} \cdot \frac{(x-3)^2}{2} =
    \\ = \frac{3x+9+\bigl((x-9)(x-3)\bigr)}{2(x+3)} = \frac{3x+9+(x^2-3x-9x+27)}{2(x+3)} = \\ =
    \doubleunderline{\frac{x^2-9x+36}{2(x+3)}}
\end{gather*}
\begin{gather*}
    \frac{x^2+7x}{9-x^2} : \frac{x^2-49}{x+3} = \frac{x(x+7)}{(x+3)(-x+3)} \cdot \frac{x+3}{(x+7)(x-7)} = \doubleunderline{\frac{x}{(x-7)(3-x)}}
\end{gather*}

\begin{align*}
    f(a) &= \frac{2x+3}{3x+4}
     \\P_y &= \left[0,\frac{3}{4}\right]
     \\P_x &= \left[-\frac{3}{2},0\right] 
     \\ 2x+3 &= 0 
     \\ x &= - \frac{3}{2}
     \\ \\ \\ \frac{\frac{2}{3} \cdot \left(x+\frac{4}{3}\right) - \left(\frac{4}{3} - \frac{3}{2}\right) \cdot\frac{2}{3}}{x+\frac{4}{3}}
\end{align*}
\end{document}
